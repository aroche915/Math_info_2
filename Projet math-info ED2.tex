\documentclass{article}
\begin{document}

\paragraph{Question 3} 

Supposons que $f$ est $C^{1}$ \\

On calcule l'erreur locale: $e^{j+1} = (x^{j}+ \int_{t_{j}}^{t_{j+1}} f(s,x(s)) \, \mathrm{d}s ) - x^{j+1}$ \\
Pour cela, on s'intéresse à la solution du problème de Cauchy 
$\left \{
\begin{array}{rcl}
\dot x&=&f(t,x) \\
x(t_{j})&=&x^{j}
\end{array}
\right.$ \\

On a donc \\$x^{j}+ \int_{t_{j}}^{t_{j+1}} f(s,x(s)) \, \mathrm{d}s = x(t_{j+1})$\\

$f$ est $C^{1}$ donc $x$ est $C^{2}$. On peut donc écrire son développement limité à l'ordre 2: \\

$x(t_{j+1}) = x(t_{j}) + \Delta t f(t_{j}, x(t_{j})) + \frac{\Delta t^{2}} {2} \ddot{x} (t_{j}) + O(\Delta t^{3})$ \\

Or $\ddot{x} (t_{j}) = \frac{ \partial }{ \partial t} f(t_{j}, x(t_{j})) = \partial _{1}f(t_{j},x(t_{j})) + \partial _{2} f(t_{j}), x(t_{j})) \times f(t_{j}, x(t_{j}))$\\

D'où: \\

$x(t_{j+1}) = x(t_{j}) + \Delta t f(t_{j}, x(t_{j})) + \frac{\Delta t^{2}} {2}  \partial _{1} f(t_{j}, x(t_{j})) + \frac{\Delta t^{2}} {2} \partial _{2} f(t_{j}, x(t_{j})) f(t_{j}, x(t_{j})) + O(\Delta t^{3})$ \\

On utilise un schéma d'Euler explicite donc $x^{j+1} - x(t_{j}) = \Delta t f(t_{j}, x(t_{j}))$\\
On remplace:\\

$x(t_{j+1}) = x(t_{j}) + \frac{\Delta t } {2} f(t_{j}, x(t_{j})) + \frac{\Delta t} {2} \left[  f(t_{j}, x(t_{j})) + \Delta t \partial _{1} f(t_{j}, x(t_{j})) + \partial _{2} f(t_{j}, x(t_{j})) (x^{j+1} - x(t_{j}))\right] + O(\Delta t^{3})$ \\

On utilise le développement limité à l'ordre 1 de $f$, ce qui nous permet d'écrire: \\

$x(t_{j+1}) = x(t_{j}) + \frac{\Delta t} {2} \left[ f(t_{j}, x(t_{j})) + f(t_{j+1}, x(t_{j+1})) \right]  + O(\Delta t^{3}) $\\

Or $e^{j+1} = x(t_{j+1}) - x^{j+1} = x(t_{j+1})-x(t_{j}) - \Delta t f(t_{j}, x(t_{j}))$ \\

Donc finalement\\

$\Vert e^{j+1} \Vert = \Delta t \frac{\Vert f(t_{j+1}, x(t_{j+1})) - f(t_{j}, x(t_{j}))\Vert} {2} + O(\Delta t^{3}) $
\\
\\
\paragraph{Question 4}
Puisque $x'(t)$ = $f(x,t)$ :

$e^{j+1} = \int_{t_{j}}^{t_{j+1}} x'(s)ds - \Delta t_{j} (\frac{x(t_{j}+\Delta t_{j})-x(t_{j})}{\Delta t_{j}} \ 
+ O(\Delta t_{j})) $

$e^{j+1} = x(t_{j+1}) - x(t_{j}) - x(t_{j+1}) + x(t_{j}) + O(\Delta t_{j}^2)$

$e^{j+1} = O(\Delta t_{j}^2)$

Ainsi, on a d'une part, si on note $N$ la norme associée :

$N(e^{j+1}) \leq  \hspace{2mm} c_{1} \Delta t_{j}^2$

D'autre part :

$N(e^{j+2} \leq  \hspace{2mm} c_{2} \Delta t_{j+1}^2$

On souhaite avoir, de proche en proche $e^{j+1} \simeq \hspace{2mm} e^{j+2} \simeq \hspace{2mm} Tol_{abs}$  
afin que l'erreur reste bornée. On souhaite donc avoir :

$c_{1} \Delta t_{j}^2 \simeq \hspace{2mm} c_{2} \Delta t_{j+1}^2$


On peut donc choisir le rapport entre $c_{1}$ et $c_{2}$ afin de valider la condition exposée plus haut. 
On peut le prendre égal à :

$\frac{Tol_{abs}}{N(e^{j+1})} \ $

On obtient alors, par passage à la racine, le $\Delta t_{new} \hspace{2mm}$
proposé dans l'énoncé.



\end{document}